\section*{Results}
\subsection*{Comparison with existing methods}
Using previously published mutational catalogues from 119 breast cancer genomes and 88 liver cancer genomes \cite{Alexandrov2013}, we benchmarked sigfit against the software tools that originally introduced the two signature formulations implemented in the package: WTSI-NMF \cite{AlexandrovCellRep2013} and EMu \cite{Fischer2013} (\hyperref[figure3]{Fig. \ref*{figure3}A--E}).

When comparing the Dirichlet–Multinomial and Poisson signature extraction models in sigfit to the original WTSI-NMF and EMu methods, we found that the approaches for estimating the most likely number of signatures in the latter two tools did not agree in any of the data sets. For the set of 119 breast cancer catalogues, sigfit (both models) and WTSI-NMF estimated the most likely number of signatures to be $N=6$, whereas EMu estimated $N=5$ as the best value. Conversely, for the set of 88 liver cancer catalogues, sigfit (both models) and EMu estimated the best number of signatures to be $N=6$, whereas WTSI-NMF estimated it as $N=10$. The contrast arises from the different approaches followed by each tool: while sigfit selects the number of signatures that minimises the second derivative of the overall cosine similarity between the original and reconstructed catalogues (essentially the `elbow method', used in the clustering literature as a heuristic to estimate the number of clusters in a data set; see example in \hyperref[figure2]{Fig. \ref*{figure2}D}), WTSI-NMF minimises the Frobenius reconstruction error while maximising signature reproducibility \cite{AlexandrovCellRep2013}, and EMu uses the Bayesian information criterion (BIC) to correct for increased model complexity \cite{Fischer2013}. The fact that sigfit consistently estimated the same optimal number of signatures for the Dirichlet–Multinomial and Poisson models, while inferring highly similar signatures and exposures to those obtained by the other methods (\hyperref[figure3]{Fig. \ref*{figure3}A--C}), suggests that the criterion followed by sigfit to estimate the most plausible number of signatures may be more robust.

Notably, EMu estimates BIC values and selects the optimal number of signatures prior to signature extraction, whereas both WTSI-NMF and sigfit perform extraction for different numbers of signatures and measure the goodness of fit of each solution. Together with the fact that the expectation–maximisation algorithm is not guaranteed to converge to a global maximum of the likelihood \cite{Dempster1977}, this can result in cases where EMu converges to a local, but not global, maximum for some values of $N$, resulting in incorrect estimation of the number of signatures that best explain the data. This is exemplified in the set of 119 breast cancer catalogues, where EMu converged to a locally optimal solution for $N=6$ signatures (\hyperref[figure3]{Fig. \ref*{figure3}D}), which, as mentioned above, led it to propose $N=5$ as the best number of signatures. Interestingly, the difference between the solutions reported by sigfit and EMu for $N=6$ is more prominent in terms of the inferred exposures than in the inferred signatures (\hyperref[figure3]{Fig. \ref*{figure3}A, C}). By contrast, WTSI-NMF and both models in sigfit found virtually identical solutions for $N=6$ signatures in this data set (\hyperref[figure3]{Fig. \ref*{figure3}A, C}), which were considerably better than the one reported by EMu in terms of reconstruction accuracy (\hyperref[figure3]{Fig. \ref*{figure3}B}). When applied to the set of 88 liver cancer catalogues, however, all the methods showed remarkably similar reconstruction accuracy distributions (\hyperref[figure3]{Fig. \ref*{figure3}B}).

It is worth noting that, whereas EMu performs efficient maximum-likelihood estimation via EM \cite{Fischer2013}, the WTSI-NMF method entails computing a considerable number of bootstrap replicates in order to identify stable clusters of mutational signatures \cite{AlexandrovCellRep2013}. This causes WTSI-NMF to be very computationally expensive, and probably best suited for highly parallel computing infrastructures. By contrast, the models in sigfit, which exploit the efficient No-U-Turn-Sampler algorithm for MCMC sampling implemented in the Stan framework \citep{stanMain}, incur only moderate memory and CPU demands that are easily met by laptop or desktop computers, while providing virtually identical solutions to those obtained by WTSI-NMF (\hyperref[figure3]{Fig. \ref*{figure3}A--C}).

In addition to its favourable performance in signature extraction problems, when fitting the entire set of 30 COSMIC signatures to a previously published set of catalogues from 21 breast cancer genomes \cite{Nik-Zainal2012:mp21bc}, sigfit reported only six significantly active signatures (\hyperref[figure3]{Fig. \ref*{figure3}E}), five of which have been previously described to be active in breast cancer \cite{Nik-Zainal2016, Alexandrov2013}. By contrast, optimisation-based methods such as least-squares regression and quadratic programming are notoriously prone to overfitting \cite{Harrell2014}, especially when applied to non-convex functions like those involved in signature extraction and fitting problems. Hence, we believe that sigfit's capability to estimate realistic contributions of known signatures to sets of catalogues (or even to single catalogues), without the need to constrain the input set of candidate signatures to prevent overfitting, makes the package valuable for small-sized genomic studies with insufficient statistical power for signature extraction.

\subsection*{Fit-Ext model validation on simulated mutation data}
As discussed above, the Fit-Ext models included in sigfit can be of use in situations where there is qualitative evidence of the presence of rare or previously undescribed signatures, yet lack of statistical power precludes conventional extraction of such signatures (or even signature extraction altogether). To demonstrate the effectiveness of the combined fitting–extraction approach in such situations — specifically, in the case where signature extraction is not possible — we repeatedly applied the Dirichlet–Multinomial Fit-Ext model to individual simulated mutational catalogues, which were drawn from a multinomial distribution defined as a random mix of three COSMIC signatures (signatures 1, 2 and 7) and one simulated novel signature (\hyperref[figure4]{Fig. \ref*{figure4}A}). The novel signature was drawn from a Dirichlet distribution with very low concentration parameter values ($\alpha_1, ..., \alpha_{96}$ = 0.05), and bore no resemblance to any signature in COSMIC (cosine similarity $<0.327$). In each of 100 simulation experiments, sigfit was applied to such artificial catalogue in two ways. First, the Dirichlet–Multinomial signature fitting model was used to fit all 30 COSMIC signatures to the simulated catalogue; this resulted in inaccurate catalogue reconstruction (median cosine similarity of 0.887 between the original and reconstructed catalogues), as none of the COSMIC signatures captured the distinctive features of the simulated novel signature (\hyperref[figure4]{Fig. \ref*{figure4}B}). Secondly, the Dirichlet–Multinomial Fit-Ext model was applied to fit all COSMIC signatures to the simulated catalogue, while simultaneously extracting one additional signature. Because a single catalogue was used in each simulation, the original and reconstructed catalogues were always identical, as any potential reconstruction error arising from random sampling was incorporated into the extracted signature (\hyperref[figure4]{Fig. \ref*{figure4}B}). Despite this, the Fit-Ext results show that the simulated novel signature could be extracted with high accuracy in the majority of cases (median cosine similarity 0.948; \hyperref[figure4]{Fig. \ref*{figure4}C}). This illustrates the model's potential in scenarios where available signatures do not entirely explain the mutational patterns found in the data.