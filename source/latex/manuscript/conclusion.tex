\section*{Conclusions}
Although the number of available tools for mutational signature analysis is growing rapidly \cite{Baez-Ortega2017}, many existing methods exhibit recurrent conceptual or practical limitations, including excessive computational requirements, convergence to local optima, low user-friendliness and rigidity in the types of data and analyses supported. By exploiting the versatility of the R programming language and the robustness of the Bayesian inference machinery offered by the Stan \cite{stanMain} framework, sigfit provides new methods for flexible, simple and efficient analysis of mutational signatures. Furthermore, sigfit is, to the best of our knowledge, the first package to allow simultaneous fitting and extraction of mutational signatures, enhancing statistical power for the discovery of rare or novel signatures that cannot be deconvoluted using standard approaches. In addition, the popular R programming environment facilitates reproducibility and integration of data and results with different bioinformatic analysis workflows.